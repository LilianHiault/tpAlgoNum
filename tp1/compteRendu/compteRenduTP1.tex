\documentclass{article}

\usepackage[utf8]{inputenc}
\usepackage[T1]{fontenc}
\usepackage[english,french]{babel}
\usepackage{textcomp}
\usepackage{amsmath,amssymb}
\usepackage{lmodern}
\usepackage[a4paper]{geometry}
\usepackage{graphicx}
\usepackage{xcolor}
\usepackage{microtype}
\usepackage{lipsum}
\usepackage{moreverb}
\usepackage{hyperref}
\hypersetup{pdfstartview=XYZ}

\title{Compte rendu : TP d'algorithmes numériques 1}
\author{HIAULT Lilian, VALLET Baptiste}
\date{}

\begin{document}

\begin{figure}[t]
 \centerline{\includegraphics[scale=0.1]{logoUCA.jpg}}
\end{figure}

\maketitle

\tableofcontents

\section*{Introduction}

À l'occasion des travaux pratiques d'algorithmes numériques HIAULT Lilian et VALLET Baptiste avons réalisé un programme en langage C qui permet de résoudre des systèmes linéaires grâce aux méthodes de Gauss et de Cholesky.


\section{Rappel des méthodes}

\subsection{Méthode de Gauss}

Méthode de Gauss

\subsection{Méthode de Cholesky}

Méthode de Cholesky

\section{Présentation des programmes commentés}

\subsection{Programme de résolution par la méthode de Gauss}

\begin{boxedverbatim}
Programme Gauss
\end{boxedverbatim}

\subsection{Programme de résolution grâce à la méthode de Cholesky}

\begin{boxedverbatim}
Programme de Cholesky
\end{boxedverbatim}

\section{Jeux d'essais}

Présentation de jeux d'essais pertinents et justifiés

Jeux d'essais : matrices tests

\section{Commentaire des jeux d'essais}

Commentaire des jeux d'essais à partir de données relatives.
Pourcentage d'écart, calcul de fonction d'erreurs, vitesse de convergence, complexité pratique, ...

\section{Conclusion générale sur les méthodes}

Comparaison, cadre d'utilisation, stabilité, ...

Peut-on retrouver une solution connue à priori ?

Stabilité : le résultat est-il modifié par des calculs dégradés (erreurs accumulées...)

Conditionnement : quel est l'effet de perturbations des données ?

Evaluation des coûts en place et en temps.


\end{document}
