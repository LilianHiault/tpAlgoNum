\documentclass{article}

\usepackage[utf8]{inputenc}
\usepackage[T1]{fontenc}
\usepackage[english,french]{babel}
\usepackage{textcomp}
\usepackage{amsmath,amssymb}
\makeatletter
\renewcommand*\env@matrix[1][*\c@MaxMatrixCols c]{%
   \hskip -\arraycolsep
   \let\@ifnextchar\new@ifnextchar
   \array{#1}}
\makeatother
\usepackage{lmodern}
\usepackage[a4paper]{geometry}
\usepackage{graphicx}
\usepackage{xcolor}
\usepackage{microtype}
\usepackage{lipsum}
\usepackage{moreverb}
\usepackage{hyperref}
\hypersetup{pdfstartview=XYZ}

\title{Compte rendu : TP d'algorithmes numériques 1}
\author{HIAULT Lilian, VALLET Baptiste}
\date{08 novembre 2019}

\begin{document}

\begin{figure}[t]
 \centerline{\includegraphics[scale=0.1]{logoUCA.jpg}}
\end{figure}

\maketitle

\tableofcontents

\newpage

\section*{Introduction}

À l'occasion des travaux pratiques d'algorithmes numériques HIAULT Lilian et VALLET Baptiste avons réalisé un programme en langage C qui permet de résoudre des systèmes linéaires grâce aux méthodes de Gauss et de Cholesky.


\section{Rappel des méthodes de résolution d'équations linéaires}

Les méthode de Gauss et de Cholesky permettent de résoudre des systèmes d'équation linéaires formé de plusieurs équations linéaires.

Par exemple :

\[\begin{cases}
  2x + y = 5 \\
  -x + 3y = 1
 \end{cases}\]

On peut visualiser ce système d'inconnues $x$ et $y$ par des matrices de type $Ax=b$ :
\[\begin{pmatrix}
  2  & 1 \\
  -1 & 3
 \end{pmatrix}
 \begin{pmatrix}
  x \\
  y
 \end{pmatrix}
 =
 \begin{pmatrix}
  5 \\
  1
 \end{pmatrix}\]

Sous forme de matrice augmentée $A$ :

\[ A =
 \begin{pmatrix}[cc|c]
  2  & 1 & 5 \\
  -1 & 3 & 1
 \end{pmatrix}\]

On utilise ces matrices pour résoudre les sytèmes d'équations linéaires.

\subsection{Méthode de Gauss}

La méthode de Gauss permet de calculer des solutions exactes d'un système d'équation en un nombre d'itération fini.
À chaque étape on doit créer des $0$ en dessous de la diagonale d'un matrice $A$ jusqu'à obtenir une matrice $A'$ diagonale supérieure grâce à laquelle on pourra résoudre directement l'équation.

\[ A' =
 \begin{pmatrix}[cc|c]
  1 & \frac{1}{2} & \frac{5}{2} \\
  0 & 1           & 1
 \end{pmatrix}\]

On a donc :
\[\begin{cases}
  x + \frac{1}{2}y = \frac{5}{2} \\
  y = 1
 \end{cases}
 \iff
 \begin{cases}
  x = 2 \\
  y = 1
 \end{cases}
\]


\subsection{Méthode de Cholesky}

Méthode de Cholesky

\section{Présentation des programmes commentés}

\subsection{Programme de résolution par la méthode de Gauss}

\begin{boxedverbatim}
 Programme Gauss
\end{boxedverbatim}

\subsection{Programme de résolution grâce à la méthode de Cholesky}

\begin{boxedverbatim}
 Programme de Cholesky
\end{boxedverbatim}

\section{Jeux d'essais}

Présentation de jeux d'essais pertinents et justifiés

Jeux d'essais : matrices tests

\section{Commentaire des jeux d'essais}

Commentaire des jeux d'essais à partir de données relatives.
Pourcentage d'écart, calcul de fonction d'erreurs, vitesse de convergence, complexité pratique, ...

\section{Conclusion générale sur les méthodes}

Comparaison, cadre d'utilisation, stabilité, ...

Peut-on retrouver une solution connue à priori ?

Stabilité : le résultat est-il modifié par des calculs dégradés (erreurs accumulées...)

Conditionnement : quel est l'effet de perturbations des données ?

Evaluation des coûts en place et en temps.


\end{document}
